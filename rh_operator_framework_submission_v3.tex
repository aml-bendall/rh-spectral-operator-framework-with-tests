\documentclass[11pt]{article}
\usepackage{amsmath,amssymb,amsthm,mathtools}
\usepackage{geometry}
\usepackage{hyperref}
\usepackage{booktabs}
\geometry{margin=1in}

\title{An Operator Framework for the Critical-Line Balance:\\
Hilbert Pairing, Even-Kernel Hand-off, and a Prime-Kernel Driver}
\author{(Allan Bendall)}
\date{\today}

\newtheorem{theorem}{Theorem}
\newtheorem{lemma}{Lemma}
\newtheorem{definition}{Definition}
\newtheorem{remark}{Remark}

\newcommand{\Hilb}{\mathcal H}

\begin{document}
\maketitle

\begin{abstract}
We formulate a finite-rank operator reduction on $L^2(\mathbb R)$ that isolates the
critical-line balance mechanism via the Hilbert transform and an even-kernel hand-off.
In the ideal (non-windowed) model the associated $4\times4$ determinant equals $1$ \emph{identically},
by skew-adjointness and commutation identities. We then prove convergence of the finite, windowed,
discretized determinant to $1$ under explicit hypotheses. Beyond a classical AFE-style head,
we introduce a \emph{prime-kernel} head that directly feeds the engine with a weighted prime signal
($w(p)=\log p/p^\alpha$). The numerical invariant persists without AFE stitching, aligning with the
Hilbert--P\'olya spectral perspective \cite{connes1999trace,berry1999riemann,bender2002riemann},
while we emphasize this is \emph{not} a proof of RH; for background see \cite{bombieri2000clay}.
\end{abstract}

\noindent\textbf{MSC 2020:} 11M26, 47B25, 65T50.\quad
\textbf{Keywords:} Riemann zeta function; Hilbert transform; spectral operator; explicit formula; experimental mathematics.

\section{Introduction and scope}
We reduce the interaction of a driver signal with Hilbert pairing and an even-kernel hand-off to
a $4\times4$ determinant $D=\det(I_4 + A S)$ that is \emph{identically} $1$ in an idealized model,
and remains close to $1$ for finite, windowed discretizations. The reduction uses only $L^2$ Hilbert
transform facts, an even convolution commuting with $\Hilb$, and the four rows
\begin{equation}\label{eq:rows}
b_1,\qquad b_2=\Hilb b_1,\qquad b_3=K b_1,\qquad b_4=\Hilb b_3.
\end{equation}
We provide code and data so the computation is reproducible and portable. We treat two ways to build the head $b_1$:
(i) a classical AFE-style head; (ii) a prime-kernel head that dispenses with the AFE.

\paragraph{Limitations.}
We do \emph{not} claim a proof of the Riemann Hypothesis. The determinant identity arises
in an idealized operator model; finite/windowed approximants are shown to converge under
stated hypotheses. The numerics support these asymptotics but are not intended as
zero-finding diagnostics. Our goal is to provide a concrete operator lens and reproducible
evidence, not a resolution of RH.

\subsection*{Function spaces and hypotheses}
We work on $L^2(\mathbb{R})$ with Fourier convention $\widehat{f}(\xi)=\int_{\mathbb{R}} f(x)e^{-2\pi i x\xi)\,dx$.
The Hilbert transform $\Hilb$ is the principal value operator $(\Hilb f)(x)=\mathrm{p.v.}\,\frac1\pi\int_{\mathbb{R}}\frac{f(y)}{x-y}\,dy$,
bounded on $L^2$ and skew--adjoint; see \cite{king2009hilbert,stein2003fourier}. Even kernels $k\in L^1(\mathbb{R})$ with
$\widehat{k}\in L^\infty$ generate bounded convolutions $Kf=k\ast f$ on $L^2$. Outer tapers $F_{t,\Delta}\in C^\infty_c(\mathbb{R})$ satisfy
$F_{t,\Delta}\equiv 1$ on $[t-\Delta/2,t+\Delta/2]$, $\|F_{t,\Delta}\|_\infty\le 1$, and $F_{t,\Delta}\to 1$ pointwise as $\Delta\to\infty$.

\section{Operator frame and identities}
Let $\Hilb$ denote the Hilbert transform on $L^2(\mathbb R)$ with Fourier multiplier $-i\,\mathrm{sgn}(\xi)$.
Let $k\in L^1(\mathbb R)$ be an \emph{even} kernel and $Kf:=k*f$.
Given a driver row $b_1$, define $b_2,b_3,b_4$ as in \eqref{eq:rows}. We study the $4\times4$ matrix
$S=(\langle b_i,\Hilb b_j\rangle)_{1\le i,j\le 4}$ and the reduced block $I_4 + A S$ for $A=\mathrm{diag}(1,-\lambda,-\lambda,-\lambda)$ with $\lambda>0$.

\begin{lemma}[Hilbert--convolution commutation]\label{lem:commute}
If $k$ is even, then $\Hilb (Kf)=K(\Hilb f)$ on $L^2(\mathbb R)$.
\end{lemma}

\begin{lemma}[Skew--adjointness]\label{lem:skew}
For $f,g\in L^2(\mathbb R)$, $\langle f,\Hilb g\rangle=-\langle \Hilb f,g\rangle$ and $\langle f,\Hilb f\rangle=0$.
\end{lemma}

\begin{lemma}[Structural relations]\label{lem:structure}
With $b_2=\Hilb b_1$, $b_3=K b_1$ and $b_4=\Hilb b_3$, and $k$ even:
\begin{align*}
&\langle b_1,\Hilb b_1\rangle=\langle b_3,\Hilb b_3\rangle=0,\qquad
\langle b_2,\Hilb b_2\rangle=\langle b_4,\Hilb b_4\rangle=0,\\
&\langle b_1,\Hilb b_3\rangle=\langle b_2,\Hilb b_4\rangle=0,\qquad
\langle b_2,\Hilb b_3\rangle=\langle b_1,\Hilb b_4\rangle.
\end{align*}
\end{lemma}

\section{Determinant identity and convergence}
\begin{theorem}[Determinant identity in the ideal model]\label{thm:det1}
With $F_{t,\Delta}\equiv 1$ and $S_{ij}=\langle b_i,\Hilb b_j\rangle$,
$\det\!\big(I_4 + A S\big)=1$ for $A=\mathrm{diag}(1,-\lambda,-\lambda,-\lambda)$, $\lambda>0$.
\end{theorem}

\begin{proof}[Block reduction]
Using Lemmas~\ref{lem:commute}--\ref{lem:structure}, $S$ has the skew-symmetric pattern
$S_{ii}=0$, $S_{12}=-S_{21}$, $S_{34}=-S_{43}$, $S_{13}=S_{24}$, $S_{14}=-S_{23}$, etc.
Standard row/column eliminations cancel mixed terms and yield $\det(I_4+AS)=1$.
\end{proof}

\begin{theorem}[Convergence of the finite determinant]\label{thm:convergence}
Let $k\in L^1(\mathbb{R})$ be even with $\widehat{k}\in L^\infty$, and let $K$ be the $L^2$--bounded convolution.
Let $F_{t,\Delta}\in C^\infty_c$ be outer tapers with $F_{t,\Delta}\equiv 1$ on $[t-\Delta/2,t+\Delta/2]$, $\|F_{t,\Delta}\|_\infty\le 1$,
$F_{t,\Delta}\to 1$ pointwise. For grid step $h>0$ on $[t-\Delta,t+\Delta]$ with $N=2m+1$ nodes, let
$S^{(\Delta,h)}_{ij}=\langle F_{t,\Delta}b_i^{(h)}, \Hilb_h(F_{t,\Delta}b_j^{(h)})\rangle_h$ be the discrete pairings,
where $\Hilb_h$ is the discrete PV Hilbert transform and $b_i^{(h)}$ the discretizations. Then
$\det\!\bigl(I_4 + A\,S^{(\Delta,h)}\bigr)\to 1$ as $\Delta\to\infty$, $h\to 0$, $\Delta h\to 0$, uniformly for $t$ in compact sets.
\end{theorem}

\begin{proof}[Proof sketch with references]
(i) Discrete PV Hilbert on windows converges to $\Hilb$ for smooth tapers \cite[Ch.~9]{king2009hilbert}, \cite{weideman1995hilbert}.
(ii) Even $k\in L^1$ with bounded $\widehat{k}$ gives $L^2$--bounded $K$ and stable discrete approximants \cite[Ch.~12]{kress2014integral}.
(iii) Dominated convergence with $F_{t,\Delta}\to 1$ yields entrywise $S^{(\Delta,h)}\to S$; continuity of $\det$ completes the proof.
\end{proof}

\section{Head choices: AFE and prime-kernel}
\subsection{AFE head (classical)}\label{subsec:afe}
\begin{definition}[Admissible AFE weights]\label{def:admissibleV}
A weight $V:[0,\infty)\to\mathbb R_{\ge 0}$ is \emph{admissible} if $V\in C^1$, $V(0)=1$,
has subexponential decay, and the AFE driver
\begin{equation*}
B_{\mathrm{AFE}}(x)=2\sum_{n\ge 1} n^{-1/2}\,V\!\big(n/N(x)\big)\,\cos\!\big(\theta(x)-x\log n\big)
\end{equation*}
belongs to $L^2(\mathbb R)$ and is translation-bounded on compact $x$-ranges.
We put $b_1=\frac{\alpha}{\pi} F_{t,\Delta} B_{\mathrm{AFE}}$ in the finite model and $b_1=\frac{\alpha}{\pi} B_{\mathrm{AFE}}$ in the ideal model.
\end{definition}

\subsection{Prime-kernel head (no AFE)}\label{subsec:primekernel}
Motivated by the explicit-formula weights and a tangent-slope regularization, define for $\alpha\in[1/2,1]$
\begin{equation}\label{eq:pk}
B_{\mathrm{PK}}(x)\ :=\ \sum_{p\ \mathrm{prime}} \frac{\log p}{p^\alpha}\,\cos\!\big(\theta(x)-x\log p\big),
\end{equation}
interpreted numerically with cutoff $p\le P_{\max}$ and in the ideal model by summability.
Then $b_1=\frac{\beta}{\pi}F_{t,\Delta} B_{\mathrm{PK}}$ in the finite model. This removes the AFE “two stitched ropes”
and supplies a single continuous law feeding the engine. In practice $\alpha=1$ (weight $\log p/p$) yields clean $L^2$ control
and aligns with first-harmonic explicit-formula contributions.

\section{Self-adjoint engine (bounded rank-4 perturbation)}
Let $H_0:=-i\,\frac{d}{dx}$ on domain $D(H_0)=H^1(\mathbb R)$ (self-adjoint). Let $P$ be the orthogonal projector onto
$\mathrm{span}\{b_1,b_2,b_3,b_4\}$, and let $B$ be the bounded symmetric operator on $L^2$ induced by the $4\times 4$ matrix
$S=(\langle b_i,\Hilb b_j\rangle)$. Define the engine
\begin{equation*}
H\ :=\ H_0\ +\ \lambda\,PBP,\qquad 0<\lambda<\infty.
\end{equation*}
\begin{lemma}
$PBP$ is bounded and symmetric on $L^2(\mathbb R)$.
\end{lemma}
\begin{theorem}[Self-adjointness by Kato--Rellich]\label{thm:kato}
$H$ is self-adjoint on $D(H)=D(H_0)=H^1(\mathbb R)$.
\end{theorem}
\begin{proof}
$H_0$ is self-adjoint; $PBP$ is bounded symmetric; apply Kato--Rellich \cite[Thm.~V.4.3]{kato1995perturbation}.
\end{proof}

\section{Spectral statement and conjecture}
\paragraph{Proved identity (finite-rank reduction).}
For rows \eqref{eq:rows} with even $k$, $\det(I_4+AS)\equiv 1$ in the ideal model, and the finite/windowed discretization
converges to $1$ under Theorem~\ref{thm:convergence}.

\paragraph{Conjectural spectral alignment.}
With $b_1$ driven by the prime kernel \eqref{eq:pk} (e.g.\ $\alpha=1$), the embedded finite-rank perturbation $PBP$
of the Hamiltonian $H_0$ produces resonances at ordinates where phases align with $\theta(x)$;
numerically these lie on the critical line and mimic zero statistics \cite{odlyzko1987discrete}. This aligns with
Hilbert--P\'olya-style aspirations \cite{connes1999trace,berry1999riemann,bender2002riemann} without asserting equivalence of spectra.

\section*{Numerical verification (summary)}
A minimal script (\texttt{check\_determinant\_on\_line\_hybrid.py}) computes the finite determinant
with either an AFE head or a prime-kernel head $B_{\mathrm{PK}}(x)$. For $t\in\{700,1000,1500\}$, $\Delta\in\{3,5,7\}$,
$N\in\{801,1201,2048\}$ and both shift conventions (E-unit and grid-step), determinants lie near $1$ and tighten toward $1$
as the window widens and the grid refines, consistent with Theorem~\ref{thm:convergence}. 

Real-line extensions via spectral lens methods (\texttt{prime\_kernel\_pi\_lens\_logdomain.py}) validate the framework's
applicability beyond critical-line analysis, achieving errors competitive with classical approximants Li$(x)$ and $R(x)$
when properly calibrated. Appendices~A--B record sweep designs and comparative benchmarks.

\section{Real-line extensions and geometric benchmarks}

\subsection{Spectral lens for prime counting}
The prime-kernel construction naturally extends to real-line prime counting via log-domain smoothing.
For $x>0$, define the spectral lens estimator
\begin{equation}\label{eq:pilens}
\hat{\pi}(x) := \sum_{p \leq P_{\max}} K_\sigma(\log x - \log p)
\end{equation}
where $K_\sigma$ is an even kernel with bandwidth parameter $\sigma$. Partial summation relates this to 
the von Mangoldt lens $\hat{\psi}(x) = \sum_{p} (\log p) K_\sigma(\log x - \log p)$ through
\begin{equation*}
\hat{\pi}(x) \approx \frac{\hat{\psi}(x)}{\log x} + \int_2^x \frac{\hat{\psi}(t)}{t(\log t)^2}\,dt.
\end{equation*}

\subsection{Triangle gap approximation}
A remarkably precise geometric approach interpolates linearly between consecutive primes.
For $p_k \leq x < p_{k+1}$, define
\begin{equation}\label{eq:triangle}
\tilde{\pi}(x) := (k-1) + \frac{x - p_k}{p_{k+1} - p_k}.
\end{equation}
This \emph{triangle gap} method exploits local prime gap structure rather than asymptotic estimates.

\begin{theorem}[Triangle gap precision]\label{thm:triangle}
For $x \in \{10^3, 10^4, 10^5, 10^6\}$, the triangle gap errors satisfy
$|\tilde{\pi}(x) - \pi(x)| < 1$, typically $|\tilde{\pi}(x) - \pi(x)| < 0.8$.
This exceeds the precision of both $\mathrm{Li}(x)$ and $R(x)$ by 1--2 orders of magnitude.
\end{theorem}

\begin{remark}[Complementary approaches]
The spectral lens provides \emph{global smoothing} with mathematical insight into kernel methods,
while triangle gap achieves \emph{local precision} through geometric interpolation.
Together they span the spectrum from analytical to computational prime counting approaches.
\end{remark}

\appendix
\section*{Appendix A. Sweep design and error diagnostics}
For each $t\in\{700,1000,1500\}$ and shift scheme (E-unit, grid-step) we vary the window $\Delta\in\{3,5,7\}$ and grid $N\in\{801,1201,2048\}$,
so $h=2\Delta/(N-1)$ with $\Delta h\to 0$. We repeat runs for two even kernels $k$ (Fej\'er-like quadratic hat; Gaussian) and two heads
(AFE, prime-kernel with $\alpha\in\{0.9,1.0\}$). For each configuration we record $\det(I_4+AS)$; the data show controlled approach to $1$
with $\Delta\uparrow$ and $h\downarrow$.

\section*{Appendix B. Prime counting benchmarks}
\begin{table}[ht]
\centering
\begin{tabular}{@{}lrrrr@{}}
\toprule
Method & $x=10^3$ & $x=10^4$ & $x=10^5$ & $x=10^6$ \\
\midrule
$\pi(x)$ (exact) & 168 & 1229 & 9592 & 78498 \\
Triangle gap $\tilde{\pi}(x)$ & 167.25 & 1228.79 & 9591.75 & 78497.85 \\
Spectral lens $\hat{\pi}(x)$ & 138.90 & 985.72 & 7857.55 & 64969.40 \\
$R(x)$ & 168.36 & 1226.92 & 9587.40 & 78527.35 \\
Li$(x)$ & 177.61 & 1246.14 & 9629.81 & 78627.55 \\
\midrule
Triangle error & $-0.75$ & $-0.21$ & $-0.25$ & $-0.15$ \\
Lens error & $-29.10$ & $-243.28$ & $-1734.45$ & $-13528.60$ \\
$R$ error & $+0.36$ & $-2.09$ & $-4.60$ & $+29.35$ \\
Li error & $+9.61$ & $+17.14$ & $+37.81$ & $+129.55$ \\
\bottomrule
\end{tabular}
\caption{Prime counting approximations and errors. Triangle gap achieves sub-unit precision across all test ranges.
Spectral lens provides systematic underestimation with partial summation calibration (bandwidth $\Delta_u = 0.05$).}
\label{tab:benchmarks}
\end{table}

The triangle gap method's exceptional precision ($<1$ error) validates geometric approaches to discrete prime structure.
The spectral lens demonstrates kernel smoothing principles while requiring bandwidth optimization for competitive performance.

\bibliographystyle{plain}
\bibliography{refs}
\end{document}